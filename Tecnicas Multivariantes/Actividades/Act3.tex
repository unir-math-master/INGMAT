\documentclass[11pt]{article}

    \usepackage[breakable]{tcolorbox}
    \usepackage{parskip} % Stop auto-indenting (to mimic markdown behaviour)
    
    \usepackage{iftex}
    \ifPDFTeX
    	\usepackage[T1]{fontenc}
    	\usepackage{mathpazo}
    \else
    	\usepackage{fontspec}
    \fi

    % Basic figure setup, for now with no caption control since it's done
    % automatically by Pandoc (which extracts ![](path) syntax from Markdown).
    \usepackage{graphicx}
    % Maintain compatibility with old templates. Remove in nbconvert 6.0
    \let\Oldincludegraphics\includegraphics
    % Ensure that by default, figures have no caption (until we provide a
    % proper Figure object with a Caption API and a way to capture that
    % in the conversion process - todo).
    \usepackage{caption}
    \DeclareCaptionFormat{nocaption}{}
    \captionsetup{format=nocaption,aboveskip=0pt,belowskip=0pt}

    \usepackage{float}
    \floatplacement{figure}{H} % forces figures to be placed at the correct location
    \usepackage{xcolor} % Allow colors to be defined
    \usepackage{enumerate} % Needed for markdown enumerations to work
    \usepackage{geometry} % Used to adjust the document margins
    \usepackage{amsmath} % Equations
    \usepackage{amssymb} % Equations
    \usepackage{textcomp} % defines textquotesingle
    % Hack from http://tex.stackexchange.com/a/47451/13684:
    \AtBeginDocument{%
        \def\PYZsq{\textquotesingle}% Upright quotes in Pygmentized code
    }
    \usepackage{upquote} % Upright quotes for verbatim code
    \usepackage{eurosym} % defines \euro
    \usepackage[mathletters]{ucs} % Extended unicode (utf-8) support
    \usepackage{fancyvrb} % verbatim replacement that allows latex
    \usepackage{grffile} % extends the file name processing of package graphics 
                         % to support a larger range
    \makeatletter % fix for old versions of grffile with XeLaTeX
    \@ifpackagelater{grffile}{2019/11/01}
    {
      % Do nothing on new versions
    }
    {
      \def\Gread@@xetex#1{%
        \IfFileExists{"\Gin@base".bb}%
        {\Gread@eps{\Gin@base.bb}}%
        {\Gread@@xetex@aux#1}%
      }
    }
    \makeatother
    \usepackage[Export]{adjustbox} % Used to constrain images to a maximum size
    \adjustboxset{max size={0.9\linewidth}{0.9\paperheight}}

    % The hyperref package gives us a pdf with properly built
    % internal navigation ('pdf bookmarks' for the table of contents,
    % internal cross-reference links, web links for URLs, etc.)
    \usepackage{hyperref}
    % The default LaTeX title has an obnoxious amount of whitespace. By default,
    % titling removes some of it. It also provides customization options.
    \usepackage{titling}
    \usepackage{longtable} % longtable support required by pandoc >1.10
    \usepackage{booktabs}  % table support for pandoc > 1.12.2
    \usepackage[inline]{enumitem} % IRkernel/repr support (it uses the enumerate* environment)
    \usepackage[normalem]{ulem} % ulem is needed to support strikethroughs (\sout)
                                % normalem makes italics be italics, not underlines
    \usepackage{mathrsfs}
    

    
    % Colors for the hyperref package
    \definecolor{urlcolor}{rgb}{0,.145,.698}
    \definecolor{linkcolor}{rgb}{.71,0.21,0.01}
    \definecolor{citecolor}{rgb}{.12,.54,.11}

    % ANSI colors
    \definecolor{ansi-black}{HTML}{3E424D}
    \definecolor{ansi-black-intense}{HTML}{282C36}
    \definecolor{ansi-red}{HTML}{E75C58}
    \definecolor{ansi-red-intense}{HTML}{B22B31}
    \definecolor{ansi-green}{HTML}{00A250}
    \definecolor{ansi-green-intense}{HTML}{007427}
    \definecolor{ansi-yellow}{HTML}{DDB62B}
    \definecolor{ansi-yellow-intense}{HTML}{B27D12}
    \definecolor{ansi-blue}{HTML}{208FFB}
    \definecolor{ansi-blue-intense}{HTML}{0065CA}
    \definecolor{ansi-magenta}{HTML}{D160C4}
    \definecolor{ansi-magenta-intense}{HTML}{A03196}
    \definecolor{ansi-cyan}{HTML}{60C6C8}
    \definecolor{ansi-cyan-intense}{HTML}{258F8F}
    \definecolor{ansi-white}{HTML}{C5C1B4}
    \definecolor{ansi-white-intense}{HTML}{A1A6B2}
    \definecolor{ansi-default-inverse-fg}{HTML}{FFFFFF}
    \definecolor{ansi-default-inverse-bg}{HTML}{000000}

    % common color for the border for error outputs.
    \definecolor{outerrorbackground}{HTML}{FFDFDF}

    % commands and environments needed by pandoc snippets
    % extracted from the output of `pandoc -s`
    \providecommand{\tightlist}{%
      \setlength{\itemsep}{0pt}\setlength{\parskip}{0pt}}
    \DefineVerbatimEnvironment{Highlighting}{Verbatim}{commandchars=\\\{\}}
    % Add ',fontsize=\small' for more characters per line
    \newenvironment{Shaded}{}{}
    \newcommand{\KeywordTok}[1]{\textcolor[rgb]{0.00,0.44,0.13}{\textbf{{#1}}}}
    \newcommand{\DataTypeTok}[1]{\textcolor[rgb]{0.56,0.13,0.00}{{#1}}}
    \newcommand{\DecValTok}[1]{\textcolor[rgb]{0.25,0.63,0.44}{{#1}}}
    \newcommand{\BaseNTok}[1]{\textcolor[rgb]{0.25,0.63,0.44}{{#1}}}
    \newcommand{\FloatTok}[1]{\textcolor[rgb]{0.25,0.63,0.44}{{#1}}}
    \newcommand{\CharTok}[1]{\textcolor[rgb]{0.25,0.44,0.63}{{#1}}}
    \newcommand{\StringTok}[1]{\textcolor[rgb]{0.25,0.44,0.63}{{#1}}}
    \newcommand{\CommentTok}[1]{\textcolor[rgb]{0.38,0.63,0.69}{\textit{{#1}}}}
    \newcommand{\OtherTok}[1]{\textcolor[rgb]{0.00,0.44,0.13}{{#1}}}
    \newcommand{\AlertTok}[1]{\textcolor[rgb]{1.00,0.00,0.00}{\textbf{{#1}}}}
    \newcommand{\FunctionTok}[1]{\textcolor[rgb]{0.02,0.16,0.49}{{#1}}}
    \newcommand{\RegionMarkerTok}[1]{{#1}}
    \newcommand{\ErrorTok}[1]{\textcolor[rgb]{1.00,0.00,0.00}{\textbf{{#1}}}}
    \newcommand{\NormalTok}[1]{{#1}}
    
    % Additional commands for more recent versions of Pandoc
    \newcommand{\ConstantTok}[1]{\textcolor[rgb]{0.53,0.00,0.00}{{#1}}}
    \newcommand{\SpecialCharTok}[1]{\textcolor[rgb]{0.25,0.44,0.63}{{#1}}}
    \newcommand{\VerbatimStringTok}[1]{\textcolor[rgb]{0.25,0.44,0.63}{{#1}}}
    \newcommand{\SpecialStringTok}[1]{\textcolor[rgb]{0.73,0.40,0.53}{{#1}}}
    \newcommand{\ImportTok}[1]{{#1}}
    \newcommand{\DocumentationTok}[1]{\textcolor[rgb]{0.73,0.13,0.13}{\textit{{#1}}}}
    \newcommand{\AnnotationTok}[1]{\textcolor[rgb]{0.38,0.63,0.69}{\textbf{\textit{{#1}}}}}
    \newcommand{\CommentVarTok}[1]{\textcolor[rgb]{0.38,0.63,0.69}{\textbf{\textit{{#1}}}}}
    \newcommand{\VariableTok}[1]{\textcolor[rgb]{0.10,0.09,0.49}{{#1}}}
    \newcommand{\ControlFlowTok}[1]{\textcolor[rgb]{0.00,0.44,0.13}{\textbf{{#1}}}}
    \newcommand{\OperatorTok}[1]{\textcolor[rgb]{0.40,0.40,0.40}{{#1}}}
    \newcommand{\BuiltInTok}[1]{{#1}}
    \newcommand{\ExtensionTok}[1]{{#1}}
    \newcommand{\PreprocessorTok}[1]{\textcolor[rgb]{0.74,0.48,0.00}{{#1}}}
    \newcommand{\AttributeTok}[1]{\textcolor[rgb]{0.49,0.56,0.16}{{#1}}}
    \newcommand{\InformationTok}[1]{\textcolor[rgb]{0.38,0.63,0.69}{\textbf{\textit{{#1}}}}}
    \newcommand{\WarningTok}[1]{\textcolor[rgb]{0.38,0.63,0.69}{\textbf{\textit{{#1}}}}}
    
    
    % Define a nice break command that doesn't care if a line doesn't already
    % exist.
    \def\br{\hspace*{\fill} \\* }
    % Math Jax compatibility definitions
    \def\gt{>}
    \def\lt{<}
    \let\Oldtex\TeX
    \let\Oldlatex\LaTeX
    \renewcommand{\TeX}{\textrm{\Oldtex}}
    \renewcommand{\LaTeX}{\textrm{\Oldlatex}}
    % Document parameters
    % Document title
    \title{Act3}
    
    
    
    
    
% Pygments definitions
\makeatletter
\def\PY@reset{\let\PY@it=\relax \let\PY@bf=\relax%
    \let\PY@ul=\relax \let\PY@tc=\relax%
    \let\PY@bc=\relax \let\PY@ff=\relax}
\def\PY@tok#1{\csname PY@tok@#1\endcsname}
\def\PY@toks#1+{\ifx\relax#1\empty\else%
    \PY@tok{#1}\expandafter\PY@toks\fi}
\def\PY@do#1{\PY@bc{\PY@tc{\PY@ul{%
    \PY@it{\PY@bf{\PY@ff{#1}}}}}}}
\def\PY#1#2{\PY@reset\PY@toks#1+\relax+\PY@do{#2}}

\expandafter\def\csname PY@tok@w\endcsname{\def\PY@tc##1{\textcolor[rgb]{0.73,0.73,0.73}{##1}}}
\expandafter\def\csname PY@tok@c\endcsname{\let\PY@it=\textit\def\PY@tc##1{\textcolor[rgb]{0.25,0.50,0.50}{##1}}}
\expandafter\def\csname PY@tok@cp\endcsname{\def\PY@tc##1{\textcolor[rgb]{0.74,0.48,0.00}{##1}}}
\expandafter\def\csname PY@tok@k\endcsname{\let\PY@bf=\textbf\def\PY@tc##1{\textcolor[rgb]{0.00,0.50,0.00}{##1}}}
\expandafter\def\csname PY@tok@kp\endcsname{\def\PY@tc##1{\textcolor[rgb]{0.00,0.50,0.00}{##1}}}
\expandafter\def\csname PY@tok@kt\endcsname{\def\PY@tc##1{\textcolor[rgb]{0.69,0.00,0.25}{##1}}}
\expandafter\def\csname PY@tok@o\endcsname{\def\PY@tc##1{\textcolor[rgb]{0.40,0.40,0.40}{##1}}}
\expandafter\def\csname PY@tok@ow\endcsname{\let\PY@bf=\textbf\def\PY@tc##1{\textcolor[rgb]{0.67,0.13,1.00}{##1}}}
\expandafter\def\csname PY@tok@nb\endcsname{\def\PY@tc##1{\textcolor[rgb]{0.00,0.50,0.00}{##1}}}
\expandafter\def\csname PY@tok@nf\endcsname{\def\PY@tc##1{\textcolor[rgb]{0.00,0.00,1.00}{##1}}}
\expandafter\def\csname PY@tok@nc\endcsname{\let\PY@bf=\textbf\def\PY@tc##1{\textcolor[rgb]{0.00,0.00,1.00}{##1}}}
\expandafter\def\csname PY@tok@nn\endcsname{\let\PY@bf=\textbf\def\PY@tc##1{\textcolor[rgb]{0.00,0.00,1.00}{##1}}}
\expandafter\def\csname PY@tok@ne\endcsname{\let\PY@bf=\textbf\def\PY@tc##1{\textcolor[rgb]{0.82,0.25,0.23}{##1}}}
\expandafter\def\csname PY@tok@nv\endcsname{\def\PY@tc##1{\textcolor[rgb]{0.10,0.09,0.49}{##1}}}
\expandafter\def\csname PY@tok@no\endcsname{\def\PY@tc##1{\textcolor[rgb]{0.53,0.00,0.00}{##1}}}
\expandafter\def\csname PY@tok@nl\endcsname{\def\PY@tc##1{\textcolor[rgb]{0.63,0.63,0.00}{##1}}}
\expandafter\def\csname PY@tok@ni\endcsname{\let\PY@bf=\textbf\def\PY@tc##1{\textcolor[rgb]{0.60,0.60,0.60}{##1}}}
\expandafter\def\csname PY@tok@na\endcsname{\def\PY@tc##1{\textcolor[rgb]{0.49,0.56,0.16}{##1}}}
\expandafter\def\csname PY@tok@nt\endcsname{\let\PY@bf=\textbf\def\PY@tc##1{\textcolor[rgb]{0.00,0.50,0.00}{##1}}}
\expandafter\def\csname PY@tok@nd\endcsname{\def\PY@tc##1{\textcolor[rgb]{0.67,0.13,1.00}{##1}}}
\expandafter\def\csname PY@tok@s\endcsname{\def\PY@tc##1{\textcolor[rgb]{0.73,0.13,0.13}{##1}}}
\expandafter\def\csname PY@tok@sd\endcsname{\let\PY@it=\textit\def\PY@tc##1{\textcolor[rgb]{0.73,0.13,0.13}{##1}}}
\expandafter\def\csname PY@tok@si\endcsname{\let\PY@bf=\textbf\def\PY@tc##1{\textcolor[rgb]{0.73,0.40,0.53}{##1}}}
\expandafter\def\csname PY@tok@se\endcsname{\let\PY@bf=\textbf\def\PY@tc##1{\textcolor[rgb]{0.73,0.40,0.13}{##1}}}
\expandafter\def\csname PY@tok@sr\endcsname{\def\PY@tc##1{\textcolor[rgb]{0.73,0.40,0.53}{##1}}}
\expandafter\def\csname PY@tok@ss\endcsname{\def\PY@tc##1{\textcolor[rgb]{0.10,0.09,0.49}{##1}}}
\expandafter\def\csname PY@tok@sx\endcsname{\def\PY@tc##1{\textcolor[rgb]{0.00,0.50,0.00}{##1}}}
\expandafter\def\csname PY@tok@m\endcsname{\def\PY@tc##1{\textcolor[rgb]{0.40,0.40,0.40}{##1}}}
\expandafter\def\csname PY@tok@gh\endcsname{\let\PY@bf=\textbf\def\PY@tc##1{\textcolor[rgb]{0.00,0.00,0.50}{##1}}}
\expandafter\def\csname PY@tok@gu\endcsname{\let\PY@bf=\textbf\def\PY@tc##1{\textcolor[rgb]{0.50,0.00,0.50}{##1}}}
\expandafter\def\csname PY@tok@gd\endcsname{\def\PY@tc##1{\textcolor[rgb]{0.63,0.00,0.00}{##1}}}
\expandafter\def\csname PY@tok@gi\endcsname{\def\PY@tc##1{\textcolor[rgb]{0.00,0.63,0.00}{##1}}}
\expandafter\def\csname PY@tok@gr\endcsname{\def\PY@tc##1{\textcolor[rgb]{1.00,0.00,0.00}{##1}}}
\expandafter\def\csname PY@tok@ge\endcsname{\let\PY@it=\textit}
\expandafter\def\csname PY@tok@gs\endcsname{\let\PY@bf=\textbf}
\expandafter\def\csname PY@tok@gp\endcsname{\let\PY@bf=\textbf\def\PY@tc##1{\textcolor[rgb]{0.00,0.00,0.50}{##1}}}
\expandafter\def\csname PY@tok@go\endcsname{\def\PY@tc##1{\textcolor[rgb]{0.53,0.53,0.53}{##1}}}
\expandafter\def\csname PY@tok@gt\endcsname{\def\PY@tc##1{\textcolor[rgb]{0.00,0.27,0.87}{##1}}}
\expandafter\def\csname PY@tok@err\endcsname{\def\PY@bc##1{\setlength{\fboxsep}{0pt}\fcolorbox[rgb]{1.00,0.00,0.00}{1,1,1}{\strut ##1}}}
\expandafter\def\csname PY@tok@kc\endcsname{\let\PY@bf=\textbf\def\PY@tc##1{\textcolor[rgb]{0.00,0.50,0.00}{##1}}}
\expandafter\def\csname PY@tok@kd\endcsname{\let\PY@bf=\textbf\def\PY@tc##1{\textcolor[rgb]{0.00,0.50,0.00}{##1}}}
\expandafter\def\csname PY@tok@kn\endcsname{\let\PY@bf=\textbf\def\PY@tc##1{\textcolor[rgb]{0.00,0.50,0.00}{##1}}}
\expandafter\def\csname PY@tok@kr\endcsname{\let\PY@bf=\textbf\def\PY@tc##1{\textcolor[rgb]{0.00,0.50,0.00}{##1}}}
\expandafter\def\csname PY@tok@bp\endcsname{\def\PY@tc##1{\textcolor[rgb]{0.00,0.50,0.00}{##1}}}
\expandafter\def\csname PY@tok@fm\endcsname{\def\PY@tc##1{\textcolor[rgb]{0.00,0.00,1.00}{##1}}}
\expandafter\def\csname PY@tok@vc\endcsname{\def\PY@tc##1{\textcolor[rgb]{0.10,0.09,0.49}{##1}}}
\expandafter\def\csname PY@tok@vg\endcsname{\def\PY@tc##1{\textcolor[rgb]{0.10,0.09,0.49}{##1}}}
\expandafter\def\csname PY@tok@vi\endcsname{\def\PY@tc##1{\textcolor[rgb]{0.10,0.09,0.49}{##1}}}
\expandafter\def\csname PY@tok@vm\endcsname{\def\PY@tc##1{\textcolor[rgb]{0.10,0.09,0.49}{##1}}}
\expandafter\def\csname PY@tok@sa\endcsname{\def\PY@tc##1{\textcolor[rgb]{0.73,0.13,0.13}{##1}}}
\expandafter\def\csname PY@tok@sb\endcsname{\def\PY@tc##1{\textcolor[rgb]{0.73,0.13,0.13}{##1}}}
\expandafter\def\csname PY@tok@sc\endcsname{\def\PY@tc##1{\textcolor[rgb]{0.73,0.13,0.13}{##1}}}
\expandafter\def\csname PY@tok@dl\endcsname{\def\PY@tc##1{\textcolor[rgb]{0.73,0.13,0.13}{##1}}}
\expandafter\def\csname PY@tok@s2\endcsname{\def\PY@tc##1{\textcolor[rgb]{0.73,0.13,0.13}{##1}}}
\expandafter\def\csname PY@tok@sh\endcsname{\def\PY@tc##1{\textcolor[rgb]{0.73,0.13,0.13}{##1}}}
\expandafter\def\csname PY@tok@s1\endcsname{\def\PY@tc##1{\textcolor[rgb]{0.73,0.13,0.13}{##1}}}
\expandafter\def\csname PY@tok@mb\endcsname{\def\PY@tc##1{\textcolor[rgb]{0.40,0.40,0.40}{##1}}}
\expandafter\def\csname PY@tok@mf\endcsname{\def\PY@tc##1{\textcolor[rgb]{0.40,0.40,0.40}{##1}}}
\expandafter\def\csname PY@tok@mh\endcsname{\def\PY@tc##1{\textcolor[rgb]{0.40,0.40,0.40}{##1}}}
\expandafter\def\csname PY@tok@mi\endcsname{\def\PY@tc##1{\textcolor[rgb]{0.40,0.40,0.40}{##1}}}
\expandafter\def\csname PY@tok@il\endcsname{\def\PY@tc##1{\textcolor[rgb]{0.40,0.40,0.40}{##1}}}
\expandafter\def\csname PY@tok@mo\endcsname{\def\PY@tc##1{\textcolor[rgb]{0.40,0.40,0.40}{##1}}}
\expandafter\def\csname PY@tok@ch\endcsname{\let\PY@it=\textit\def\PY@tc##1{\textcolor[rgb]{0.25,0.50,0.50}{##1}}}
\expandafter\def\csname PY@tok@cm\endcsname{\let\PY@it=\textit\def\PY@tc##1{\textcolor[rgb]{0.25,0.50,0.50}{##1}}}
\expandafter\def\csname PY@tok@cpf\endcsname{\let\PY@it=\textit\def\PY@tc##1{\textcolor[rgb]{0.25,0.50,0.50}{##1}}}
\expandafter\def\csname PY@tok@c1\endcsname{\let\PY@it=\textit\def\PY@tc##1{\textcolor[rgb]{0.25,0.50,0.50}{##1}}}
\expandafter\def\csname PY@tok@cs\endcsname{\let\PY@it=\textit\def\PY@tc##1{\textcolor[rgb]{0.25,0.50,0.50}{##1}}}

\def\PYZbs{\char`\\}
\def\PYZus{\char`\_}
\def\PYZob{\char`\{}
\def\PYZcb{\char`\}}
\def\PYZca{\char`\^}
\def\PYZam{\char`\&}
\def\PYZlt{\char`\<}
\def\PYZgt{\char`\>}
\def\PYZsh{\char`\#}
\def\PYZpc{\char`\%}
\def\PYZdl{\char`\$}
\def\PYZhy{\char`\-}
\def\PYZsq{\char`\'}
\def\PYZdq{\char`\"}
\def\PYZti{\char`\~}
% for compatibility with earlier versions
\def\PYZat{@}
\def\PYZlb{[}
\def\PYZrb{]}
\makeatother


    % For linebreaks inside Verbatim environment from package fancyvrb. 
    \makeatletter
        \newbox\Wrappedcontinuationbox 
        \newbox\Wrappedvisiblespacebox 
        \newcommand*\Wrappedvisiblespace {\textcolor{red}{\textvisiblespace}} 
        \newcommand*\Wrappedcontinuationsymbol {\textcolor{red}{\llap{\tiny$\m@th\hookrightarrow$}}} 
        \newcommand*\Wrappedcontinuationindent {3ex } 
        \newcommand*\Wrappedafterbreak {\kern\Wrappedcontinuationindent\copy\Wrappedcontinuationbox} 
        % Take advantage of the already applied Pygments mark-up to insert 
        % potential linebreaks for TeX processing. 
        %        {, <, #, %, $, ' and ": go to next line. 
        %        _, }, ^, &, >, - and ~: stay at end of broken line. 
        % Use of \textquotesingle for straight quote. 
        \newcommand*\Wrappedbreaksatspecials {% 
            \def\PYGZus{\discretionary{\char`\_}{\Wrappedafterbreak}{\char`\_}}% 
            \def\PYGZob{\discretionary{}{\Wrappedafterbreak\char`\{}{\char`\{}}% 
            \def\PYGZcb{\discretionary{\char`\}}{\Wrappedafterbreak}{\char`\}}}% 
            \def\PYGZca{\discretionary{\char`\^}{\Wrappedafterbreak}{\char`\^}}% 
            \def\PYGZam{\discretionary{\char`\&}{\Wrappedafterbreak}{\char`\&}}% 
            \def\PYGZlt{\discretionary{}{\Wrappedafterbreak\char`\<}{\char`\<}}% 
            \def\PYGZgt{\discretionary{\char`\>}{\Wrappedafterbreak}{\char`\>}}% 
            \def\PYGZsh{\discretionary{}{\Wrappedafterbreak\char`\#}{\char`\#}}% 
            \def\PYGZpc{\discretionary{}{\Wrappedafterbreak\char`\%}{\char`\%}}% 
            \def\PYGZdl{\discretionary{}{\Wrappedafterbreak\char`\$}{\char`\$}}% 
            \def\PYGZhy{\discretionary{\char`\-}{\Wrappedafterbreak}{\char`\-}}% 
            \def\PYGZsq{\discretionary{}{\Wrappedafterbreak\textquotesingle}{\textquotesingle}}% 
            \def\PYGZdq{\discretionary{}{\Wrappedafterbreak\char`\"}{\char`\"}}% 
            \def\PYGZti{\discretionary{\char`\~}{\Wrappedafterbreak}{\char`\~}}% 
        } 
        % Some characters . , ; ? ! / are not pygmentized. 
        % This macro makes them "active" and they will insert potential linebreaks 
        \newcommand*\Wrappedbreaksatpunct {% 
            \lccode`\~`\.\lowercase{\def~}{\discretionary{\hbox{\char`\.}}{\Wrappedafterbreak}{\hbox{\char`\.}}}% 
            \lccode`\~`\,\lowercase{\def~}{\discretionary{\hbox{\char`\,}}{\Wrappedafterbreak}{\hbox{\char`\,}}}% 
            \lccode`\~`\;\lowercase{\def~}{\discretionary{\hbox{\char`\;}}{\Wrappedafterbreak}{\hbox{\char`\;}}}% 
            \lccode`\~`\:\lowercase{\def~}{\discretionary{\hbox{\char`\:}}{\Wrappedafterbreak}{\hbox{\char`\:}}}% 
            \lccode`\~`\?\lowercase{\def~}{\discretionary{\hbox{\char`\?}}{\Wrappedafterbreak}{\hbox{\char`\?}}}% 
            \lccode`\~`\!\lowercase{\def~}{\discretionary{\hbox{\char`\!}}{\Wrappedafterbreak}{\hbox{\char`\!}}}% 
            \lccode`\~`\/\lowercase{\def~}{\discretionary{\hbox{\char`\/}}{\Wrappedafterbreak}{\hbox{\char`\/}}}% 
            \catcode`\.\active
            \catcode`\,\active 
            \catcode`\;\active
            \catcode`\:\active
            \catcode`\?\active
            \catcode`\!\active
            \catcode`\/\active 
            \lccode`\~`\~ 	
        }
    \makeatother

    \let\OriginalVerbatim=\Verbatim
    \makeatletter
    \renewcommand{\Verbatim}[1][1]{%
        %\parskip\z@skip
        \sbox\Wrappedcontinuationbox {\Wrappedcontinuationsymbol}%
        \sbox\Wrappedvisiblespacebox {\FV@SetupFont\Wrappedvisiblespace}%
        \def\FancyVerbFormatLine ##1{\hsize\linewidth
            \vtop{\raggedright\hyphenpenalty\z@\exhyphenpenalty\z@
                \doublehyphendemerits\z@\finalhyphendemerits\z@
                \strut ##1\strut}%
        }%
        % If the linebreak is at a space, the latter will be displayed as visible
        % space at end of first line, and a continuation symbol starts next line.
        % Stretch/shrink are however usually zero for typewriter font.
        \def\FV@Space {%
            \nobreak\hskip\z@ plus\fontdimen3\font minus\fontdimen4\font
            \discretionary{\copy\Wrappedvisiblespacebox}{\Wrappedafterbreak}
            {\kern\fontdimen2\font}%
        }%
        
        % Allow breaks at special characters using \PYG... macros.
        \Wrappedbreaksatspecials
        % Breaks at punctuation characters . , ; ? ! and / need catcode=\active 	
        \OriginalVerbatim[#1,codes*=\Wrappedbreaksatpunct]%
    }
    \makeatother

    % Exact colors from NB
    \definecolor{incolor}{HTML}{303F9F}
    \definecolor{outcolor}{HTML}{D84315}
    \definecolor{cellborder}{HTML}{CFCFCF}
    \definecolor{cellbackground}{HTML}{F7F7F7}
    
    % prompt
    \makeatletter
    \newcommand{\boxspacing}{\kern\kvtcb@left@rule\kern\kvtcb@boxsep}
    \makeatother
    \newcommand{\prompt}[4]{
        {\ttfamily\llap{{\color{#2}[#3]:\hspace{3pt}#4}}\vspace{-\baselineskip}}
    }
    

    
    % Prevent overflowing lines due to hard-to-break entities
    \sloppy 
    % Setup hyperref package
    \hypersetup{
      breaklinks=true,  % so long urls are correctly broken across lines
      colorlinks=true,
      urlcolor=urlcolor,
      linkcolor=linkcolor,
      citecolor=citecolor,
      }
    % Slightly bigger margins than the latex defaults
    
    \geometry{verbose,tmargin=1in,bmargin=1in,lmargin=1in,rmargin=1in}
    
    

\begin{document}
    
    \maketitle
    
    

    
    \hypertarget{actividad-nuxfamero-3-tuxe9cnicas-multivariantes.}{%
\section{Actividad Número 3, Técnicas
Multivariantes.}\label{actividad-nuxfamero-3-tuxe9cnicas-multivariantes.}}

\hypertarget{muxe1ster-en-ingenieruxeda-matemuxe1tica-y-computaciuxf3n.}{%
\subsection{Máster en ingeniería matemática y
computación.}\label{muxe1ster-en-ingenieruxeda-matemuxe1tica-y-computaciuxf3n.}}

\hypertarget{jorge-augusto-balsells-orellana.}{%
\subsection{Jorge Augusto Balsells
Orellana.}\label{jorge-augusto-balsells-orellana.}}

    \begin{tcolorbox}[breakable, size=fbox, boxrule=1pt, pad at break*=1mm,colback=cellbackground, colframe=cellborder]
\prompt{In}{incolor}{1}{\boxspacing}
\begin{Verbatim}[commandchars=\\\{\}]
\PY{c+ch}{\PYZsh{}!/usr/bin/python3}
\end{Verbatim}
\end{tcolorbox}

    \hypertarget{importaciuxf3n-de-libreruxedas}{%
\subsubsection{Importación de
librerías}\label{importaciuxf3n-de-libreruxedas}}

    \begin{tcolorbox}[breakable, size=fbox, boxrule=1pt, pad at break*=1mm,colback=cellbackground, colframe=cellborder]
\prompt{In}{incolor}{2}{\boxspacing}
\begin{Verbatim}[commandchars=\\\{\}]
\PY{k+kn}{import} \PY{n+nn}{matplotlib}\PY{n+nn}{.}\PY{n+nn}{pyplot} \PY{k}{as} \PY{n+nn}{plt}
\PY{k+kn}{import} \PY{n+nn}{statsmodels}\PY{n+nn}{.}\PY{n+nn}{api} \PY{k}{as} \PY{n+nn}{sm}
\PY{k+kn}{import} \PY{n+nn}{sklearn} \PY{k}{as} \PY{n+nn}{sk}
\PY{k+kn}{import} \PY{n+nn}{numpy} \PY{k}{as} \PY{n+nn}{np}
\PY{k+kn}{import} \PY{n+nn}{pandas} \PY{k}{as} \PY{n+nn}{pd}
\PY{k+kn}{import} \PY{n+nn}{graphviz} 
\PY{k+kn}{import} \PY{n+nn}{seaborn} \PY{k}{as} \PY{n+nn}{sns}

\PY{k+kn}{from} \PY{n+nn}{sklearn}\PY{n+nn}{.}\PY{n+nn}{model\PYZus{}selection} \PY{k+kn}{import} \PY{n}{train\PYZus{}test\PYZus{}split}
\PY{k+kn}{from} \PY{n+nn}{sklearn}\PY{n+nn}{.}\PY{n+nn}{datasets} \PY{k+kn}{import} \PY{n}{make\PYZus{}classification}
\PY{k+kn}{from} \PY{n+nn}{sklearn}\PY{n+nn}{.}\PY{n+nn}{ensemble} \PY{k+kn}{import} \PY{n}{RandomForestClassifier}\PY{p}{,} \PY{n}{BaggingClassifier}\PY{p}{,} \PY{n}{GradientBoostingClassifier}
\PY{k+kn}{from} \PY{n+nn}{sklearn}\PY{n+nn}{.}\PY{n+nn}{tree} \PY{k+kn}{import} \PY{n}{DecisionTreeClassifier}
\PY{k+kn}{from} \PY{n+nn}{sklearn}\PY{n+nn}{.}\PY{n+nn}{tree} \PY{k+kn}{import} \PY{n}{export\PYZus{}graphviz}
\PY{k+kn}{from} \PY{n+nn}{sklearn}\PY{n+nn}{.}\PY{n+nn}{metrics} \PY{k+kn}{import} \PY{n}{accuracy\PYZus{}score}
\end{Verbatim}
\end{tcolorbox}

    \hypertarget{creando-dataset.}{%
\subsubsection{Creando dataset.}\label{creando-dataset.}}

\hypertarget{documento-de-identificaciuxf3n-1663153890101---modificado-26632538}{%
\subsubsection{Documento de identificación: 1663153890101
-\textgreater{} Modificado:
26632538}\label{documento-de-identificaciuxf3n-1663153890101---modificado-26632538}}

    \begin{tcolorbox}[breakable, size=fbox, boxrule=1pt, pad at break*=1mm,colback=cellbackground, colframe=cellborder]
\prompt{In}{incolor}{3}{\boxspacing}
\begin{Verbatim}[commandchars=\\\{\}]
\PY{c+c1}{\PYZsh{}DPI modificado}
\PY{n}{dpi} \PY{o}{=} \PY{l+s+s2}{\PYZdq{}}\PY{l+s+s2}{26632538}\PY{l+s+s2}{\PYZdq{}}

\PY{c+c1}{\PYZsh{}Generación de dataset}
\PY{n}{DataSet} \PY{o}{=} \PY{n}{sk}\PY{o}{.}\PY{n}{datasets}\PY{o}{.}\PY{n}{make\PYZus{}classification}\PY{p}{(}
    \PY{n}{n\PYZus{}samples} \PY{o}{=} \PY{l+m+mi}{200} \PY{o}{+} \PY{l+m+mi}{10}\PY{o}{*}\PY{n+nb}{int}\PY{p}{(}\PY{n}{dpi}\PY{p}{[}\PY{l+m+mi}{0}\PY{p}{]}\PY{p}{)}\PY{p}{,} 
    \PY{n}{n\PYZus{}features} \PY{o}{=} \PY{l+m+mi}{10} \PY{o}{+} \PY{n+nb}{int}\PY{p}{(}\PY{n}{dpi}\PY{p}{[}\PY{l+m+mi}{1}\PY{p}{]}\PY{p}{)} \PY{o}{+} \PY{n+nb}{int}\PY{p}{(}\PY{n}{dpi}\PY{p}{[}\PY{l+m+mi}{2}\PY{p}{]}\PY{p}{)}\PY{p}{,}
    \PY{n}{n\PYZus{}informative} \PY{o}{=} \PY{l+m+mi}{10} \PY{o}{+} \PY{n+nb}{int}\PY{p}{(}\PY{n}{dpi}\PY{p}{[}\PY{l+m+mi}{1}\PY{p}{]}\PY{p}{)}\PY{p}{,} 
    \PY{n}{random\PYZus{}state} \PY{o}{=} \PY{n+nb}{int}\PY{p}{(}\PY{n}{dpi}\PY{p}{)}\PY{p}{,}
    \PY{n}{shuffle}\PY{o}{=}\PY{k+kc}{False}\PY{p}{,}

    \PY{n}{n\PYZus{}redundant}\PY{o}{=}\PY{l+m+mi}{2}\PY{p}{,} 
    \PY{n}{n\PYZus{}repeated}\PY{o}{=}\PY{l+m+mi}{0}\PY{p}{,} 
    \PY{n}{n\PYZus{}classes}\PY{o}{=}\PY{l+m+mi}{2}\PY{p}{,} 
    \PY{n}{n\PYZus{}clusters\PYZus{}per\PYZus{}class}\PY{o}{=}\PY{l+m+mi}{2}\PY{p}{,} 
    \PY{n}{weights}\PY{o}{=}\PY{k+kc}{None}\PY{p}{,} 
    \PY{n}{flip\PYZus{}y}\PY{o}{=}\PY{l+m+mf}{0.01}\PY{p}{,} 
    \PY{n}{class\PYZus{}sep}\PY{o}{=}\PY{l+m+mf}{1.0}\PY{p}{,} 
    \PY{n}{hypercube}\PY{o}{=}\PY{k+kc}{True}\PY{p}{,} 
    \PY{n}{shift}\PY{o}{=}\PY{l+m+mf}{0.0}\PY{p}{,} 
    \PY{n}{scale}\PY{o}{=}\PY{l+m+mf}{1.0} 
\PY{p}{)}

\PY{c+c1}{\PYZsh{}Longitud de dataset}
\PY{n+nb}{print}\PY{p}{(}\PY{l+s+s2}{\PYZdq{}}\PY{l+s+s2}{Length: }\PY{l+s+s2}{\PYZdq{}} \PY{o}{+} \PY{n+nb}{str}\PY{p}{(}\PY{n+nb}{len}\PY{p}{(}\PY{n}{DataSet}\PY{p}{)}\PY{p}{)}\PY{p}{)}

\PY{n}{X} \PY{o}{=} \PY{n}{DataSet}\PY{p}{[}\PY{l+m+mi}{0}\PY{p}{]}
\PY{n}{Y} \PY{o}{=} \PY{n}{DataSet}\PY{p}{[}\PY{l+m+mi}{1}\PY{p}{]}

\PY{c+c1}{\PYZsh{}Shapes con el tamaño de los arrays involucrados en DataSet }
\PY{n+nb}{print}\PY{p}{(}\PY{l+s+s2}{\PYZdq{}}\PY{l+s+s2}{Generated Samples:}\PY{l+s+s2}{\PYZdq{}} \PY{o}{+} \PY{n+nb}{str}\PY{p}{(}\PY{n}{X}\PY{o}{.}\PY{n}{shape}\PY{p}{)}\PY{p}{)}
\PY{n+nb}{print}\PY{p}{(}\PY{l+s+s2}{\PYZdq{}}\PY{l+s+s2}{Integer labels:}\PY{l+s+s2}{\PYZdq{}} \PY{o}{+} \PY{n+nb}{str}\PY{p}{(}\PY{n}{Y}\PY{o}{.}\PY{n}{shape}\PY{p}{)}\PY{p}{)}
\PY{n+nb}{print}\PY{p}{(}\PY{l+s+s2}{\PYZdq{}}\PY{l+s+s2}{El conjunto de datos contiene }\PY{l+s+s2}{\PYZdq{}}\PY{o}{+}\PY{n+nb}{str}\PY{p}{(}\PY{n}{X}\PY{o}{.}\PY{n}{shape}\PY{p}{[}\PY{l+m+mi}{1}\PY{p}{]}\PY{p}{)}\PY{o}{+}\PY{l+s+s2}{\PYZdq{}}\PY{l+s+s2}{ variables predictoras, con }\PY{l+s+s2}{\PYZdq{}}\PY{o}{+}\PY{n+nb}{str}\PY{p}{(}\PY{n}{X}\PY{o}{.}\PY{n}{shape}\PY{p}{[}\PY{l+m+mi}{0}\PY{p}{]}\PY{p}{)}\PY{o}{+}\PY{l+s+s2}{\PYZdq{}}\PY{l+s+s2}{ muestras}\PY{l+s+s2}{\PYZdq{}}\PY{p}{)}

\PY{c+c1}{\PYZsh{}Agregando nombres de columnas a DataSet}
\PY{n}{Features} \PY{o}{=} \PY{p}{[}\PY{l+s+s1}{\PYZsq{}}\PY{l+s+s1}{c0}\PY{l+s+s1}{\PYZsq{}}\PY{p}{,}\PY{l+s+s1}{\PYZsq{}}\PY{l+s+s1}{c1}\PY{l+s+s1}{\PYZsq{}}\PY{p}{,}\PY{l+s+s1}{\PYZsq{}}\PY{l+s+s1}{c2}\PY{l+s+s1}{\PYZsq{}}\PY{p}{,}\PY{l+s+s1}{\PYZsq{}}\PY{l+s+s1}{c3}\PY{l+s+s1}{\PYZsq{}}\PY{p}{,}\PY{l+s+s1}{\PYZsq{}}\PY{l+s+s1}{c4}\PY{l+s+s1}{\PYZsq{}}\PY{p}{,}\PY{l+s+s1}{\PYZsq{}}\PY{l+s+s1}{c5}\PY{l+s+s1}{\PYZsq{}}\PY{p}{,}\PY{l+s+s1}{\PYZsq{}}\PY{l+s+s1}{c6}\PY{l+s+s1}{\PYZsq{}}\PY{p}{,}\PY{l+s+s1}{\PYZsq{}}\PY{l+s+s1}{c7}\PY{l+s+s1}{\PYZsq{}}\PY{p}{,}\PY{l+s+s1}{\PYZsq{}}\PY{l+s+s1}{c8}\PY{l+s+s1}{\PYZsq{}}\PY{p}{,}\PY{l+s+s1}{\PYZsq{}}\PY{l+s+s1}{c9}\PY{l+s+s1}{\PYZsq{}}\PY{p}{,}\PY{l+s+s1}{\PYZsq{}}\PY{l+s+s1}{c10}\PY{l+s+s1}{\PYZsq{}}\PY{p}{,}\PY{l+s+s1}{\PYZsq{}}\PY{l+s+s1}{c11}\PY{l+s+s1}{\PYZsq{}}\PY{p}{,}\PY{l+s+s1}{\PYZsq{}}\PY{l+s+s1}{c12}\PY{l+s+s1}{\PYZsq{}}\PY{p}{,}\PY{l+s+s1}{\PYZsq{}}\PY{l+s+s1}{c13}\PY{l+s+s1}{\PYZsq{}}\PY{p}{,}\PY{l+s+s1}{\PYZsq{}}\PY{l+s+s1}{c14}\PY{l+s+s1}{\PYZsq{}}\PY{p}{,} \PY{l+s+s1}{\PYZsq{}}\PY{l+s+s1}{c15}\PY{l+s+s1}{\PYZsq{}}\PY{p}{,}\PY{l+s+s1}{\PYZsq{}}\PY{l+s+s1}{c16}\PY{l+s+s1}{\PYZsq{}}\PY{p}{,}\PY{l+s+s1}{\PYZsq{}}\PY{l+s+s1}{c17}\PY{l+s+s1}{\PYZsq{}}\PY{p}{,}\PY{l+s+s1}{\PYZsq{}}\PY{l+s+s1}{c18}\PY{l+s+s1}{\PYZsq{}}\PY{p}{,}\PY{l+s+s1}{\PYZsq{}}\PY{l+s+s1}{c19}\PY{l+s+s1}{\PYZsq{}}\PY{p}{,}\PY{l+s+s1}{\PYZsq{}}\PY{l+s+s1}{c20}\PY{l+s+s1}{\PYZsq{}}\PY{p}{,}\PY{l+s+s1}{\PYZsq{}}\PY{l+s+s1}{c21}\PY{l+s+s1}{\PYZsq{}}\PY{p}{]}
\PY{n}{Data} \PY{o}{=} \PY{n}{pd}\PY{o}{.}\PY{n}{DataFrame}\PY{p}{(}\PY{n}{DataSet}\PY{p}{[}\PY{l+m+mi}{0}\PY{p}{]}\PY{p}{,} \PY{n}{columns} \PY{o}{=} \PY{n}{Features}\PY{p}{)}
\PY{n+nb}{print}\PY{p}{(}\PY{n}{Data}\PY{p}{)} 
\end{Verbatim}
\end{tcolorbox}

    \begin{Verbatim}[commandchars=\\\{\}]
Length: 2
Generated Samples:(220, 22)
Integer labels:(220,)
El conjunto de datos contiene 22 variables predictoras, con 220 muestras
           c0        c1        c2        c3        c4        c5        c6  \textbackslash{}
0   -0.623798  2.886848  2.289823  1.966687  5.857388  2.864425  1.240953
1   -4.230493  2.574538 -1.888288  0.034745  0.474872 -3.073048 -3.280422
2    2.403653 -0.138274  3.773707  2.264375  4.246538  0.156183  3.108223
3   -2.314204 -2.136062  0.104248  2.047772 -0.883033 -4.931123 -1.713463
4   -1.859431 -1.098524  0.552397  5.169887  1.383473 -1.257396  4.124868
..        {\ldots}       {\ldots}       {\ldots}       {\ldots}       {\ldots}       {\ldots}       {\ldots}
215 -4.483727 -2.943545  1.243704  3.460865  0.145977  1.130336 -1.093742
216 -2.571963  3.924738  3.771498 -0.270276  1.586456  2.028533  4.302808
217 -1.107106  2.239045 -2.226861 -0.669305 -2.780989  0.468113 -2.042945
218 -3.406655  3.023067 -6.719494  2.827009 -0.320336 -2.108107 -3.761495
219  4.003681 -0.878472 -2.034100 -1.018294 -2.088653  4.197321  0.759902

           c7        c8        c9  {\ldots}       c12       c13       c14  \textbackslash{}
0   -0.519994  0.466540 -2.240870  {\ldots} -5.508834 -2.004952 -4.410966
1   -5.113795 -2.293789  3.113564  {\ldots}  8.674391  2.086394  2.838027
2   -1.907037 -3.568915 -3.096199  {\ldots} -3.513964  1.267124 -4.145067
3   -0.655707 -1.854736 -0.733168  {\ldots}  3.756722  1.588267  1.103204
4    2.787510  1.222891  2.233237  {\ldots}  0.919870  3.427640  2.300415
..        {\ldots}       {\ldots}       {\ldots}  {\ldots}       {\ldots}       {\ldots}       {\ldots}
215  0.931300 -2.070931 -1.808822  {\ldots}  3.219543 -1.258340  3.718142
216  2.677844 -0.692493 -3.135414  {\ldots}  1.140616  1.785074 -3.016355
217 -0.294405  1.369012 -2.065447  {\ldots} -0.415357 -3.724343  1.699928
218  3.270016  3.113855  0.599535  {\ldots} -1.411823 -0.227754 -1.022757
219  2.902194  5.646378  1.757992  {\ldots} -0.853214 -0.563142 -0.183372

          c15        c16        c17       c18       c19       c20       c21
0   -6.332895 -13.716803   0.886687  0.303550 -0.863660  0.106751 -0.997125
1    3.845933   1.936287 -10.732193  0.045495 -0.006526 -0.447797 -1.851041
2   -3.132932  -9.629241  -0.500882  0.877463  0.713304  0.370858  0.799430
3   -1.747323  -0.558611  -0.750382 -0.251319  1.013492 -0.070628 -1.958347
4   -1.669274  -4.316897   4.477207 -0.064378  2.333631 -0.639747 -1.355723
..        {\ldots}        {\ldots}        {\ldots}       {\ldots}       {\ldots}       {\ldots}       {\ldots}
215 -0.464070  -3.788584   0.444629  0.290284  0.990517  0.993528 -1.160651
216  1.916252  -5.114002  -1.794398  1.008627 -0.199072 -0.690215  1.526858
217 -0.219343  -1.656995  -4.320898 -0.245096  2.728810  1.447858 -0.030568
218 -2.618619 -11.652146   0.006426 -3.117684  1.919126  0.496763  0.179371
219  5.115366   6.597616   7.785048 -0.655601 -0.558329 -0.012811  0.365807

[220 rows x 22 columns]
    \end{Verbatim}

    \hypertarget{divide-el-conjunto-de-datos-en-200-observaciones-para-el-entrenamiento-y-el-resto-para-realizar-la-validaciuxf3n-de-los-distintos-muxe9todos-de-regresiuxf3n-aplicados.}{%
\subsubsection{Divide el conjunto de datos en 200 observaciones para el
entrenamiento y el resto para realizar la validación de los distintos
métodos de regresión
aplicados.}\label{divide-el-conjunto-de-datos-en-200-observaciones-para-el-entrenamiento-y-el-resto-para-realizar-la-validaciuxf3n-de-los-distintos-muxe9todos-de-regresiuxf3n-aplicados.}}

    \begin{tcolorbox}[breakable, size=fbox, boxrule=1pt, pad at break*=1mm,colback=cellbackground, colframe=cellborder]
\prompt{In}{incolor}{4}{\boxspacing}
\begin{Verbatim}[commandchars=\\\{\}]
\PY{c+c1}{\PYZsh{}Separacion de datos de entrenamiento y datos de prueba}
\PY{n}{TrainSet}\PY{p}{,} \PY{n}{TestSet} \PY{o}{=} \PY{n}{train\PYZus{}test\PYZus{}split}\PY{p}{(}\PY{n}{Data}\PY{p}{,} \PY{n}{test\PYZus{}size}\PY{o}{=}\PY{l+m+mf}{0.09}\PY{p}{,} \PY{n}{random\PYZus{}state}\PY{o}{=}\PY{l+m+mi}{42}\PY{p}{)}
\PY{n}{YTrain}\PY{p}{,} \PY{n}{YTest} \PY{o}{=} \PY{n}{train\PYZus{}test\PYZus{}split}\PY{p}{(}\PY{n}{Y}\PY{p}{,} \PY{n}{test\PYZus{}size}\PY{o}{=}\PY{l+m+mf}{0.09}\PY{p}{,} \PY{n}{random\PYZus{}state}\PY{o}{=}\PY{l+m+mi}{42}\PY{p}{)}

\PY{c+c1}{\PYZsh{}Mostrar shape de datos de entrenamiento}
\PY{n+nb}{print}\PY{p}{(}\PY{l+s+s2}{\PYZdq{}}\PY{l+s+s2}{Datos de entrenamiento}\PY{l+s+s2}{\PYZdq{}}\PY{p}{)}
\PY{n+nb}{print}\PY{p}{(}\PY{n}{TrainSet}\PY{o}{.}\PY{n}{shape}\PY{p}{)}
\PY{n+nb}{print}\PY{p}{(}\PY{n}{YTrain}\PY{o}{.}\PY{n}{shape}\PY{p}{)}

\PY{c+c1}{\PYZsh{}Mostrar shape de datos de prueba}
\PY{n+nb}{print}\PY{p}{(}\PY{l+s+s2}{\PYZdq{}}\PY{l+s+s2}{Datos de prueba}\PY{l+s+s2}{\PYZdq{}}\PY{p}{)}
\PY{n+nb}{print}\PY{p}{(}\PY{n}{TestSet}\PY{o}{.}\PY{n}{shape}\PY{p}{)}
\PY{n+nb}{print}\PY{p}{(}\PY{n}{YTest}\PY{o}{.}\PY{n}{shape}\PY{p}{)}
\end{Verbatim}
\end{tcolorbox}

    \begin{Verbatim}[commandchars=\\\{\}]
Datos de entrenamiento
(200, 22)
(200,)
Datos de prueba
(20, 22)
(20,)
    \end{Verbatim}

    \hypertarget{describe-tu-conjunto-de-datos-transfuxf3rmalo-en-un-data.frame-aplica-los-muxe9todos-.info-.describe-y-obtuxe9n-el-histograma-o-diagrama-de-barras-de-todas-las-variables-predictoras-y-la-variable-respuesta.}{%
\subsubsection{Describe tu conjunto de datos (transfórmalo en un
data.frame, aplica los métodos .info(), .describe() y obtén el
histograma o diagrama de barras de todas las variables (predictoras y la
variable
respuesta).}\label{describe-tu-conjunto-de-datos-transfuxf3rmalo-en-un-data.frame-aplica-los-muxe9todos-.info-.describe-y-obtuxe9n-el-histograma-o-diagrama-de-barras-de-todas-las-variables-predictoras-y-la-variable-respuesta.}}

    \begin{tcolorbox}[breakable, size=fbox, boxrule=1pt, pad at break*=1mm,colback=cellbackground, colframe=cellborder]
\prompt{In}{incolor}{5}{\boxspacing}
\begin{Verbatim}[commandchars=\\\{\}]
\PY{n}{TrainSet}\PY{o}{.}\PY{n}{info}\PY{p}{(}\PY{p}{)}
\end{Verbatim}
\end{tcolorbox}

    \begin{Verbatim}[commandchars=\\\{\}]
<class 'pandas.core.frame.DataFrame'>
Int64Index: 200 entries, 189 to 102
Data columns (total 22 columns):
 \#   Column  Non-Null Count  Dtype
---  ------  --------------  -----
 0   c0      200 non-null    float64
 1   c1      200 non-null    float64
 2   c2      200 non-null    float64
 3   c3      200 non-null    float64
 4   c4      200 non-null    float64
 5   c5      200 non-null    float64
 6   c6      200 non-null    float64
 7   c7      200 non-null    float64
 8   c8      200 non-null    float64
 9   c9      200 non-null    float64
 10  c10     200 non-null    float64
 11  c11     200 non-null    float64
 12  c12     200 non-null    float64
 13  c13     200 non-null    float64
 14  c14     200 non-null    float64
 15  c15     200 non-null    float64
 16  c16     200 non-null    float64
 17  c17     200 non-null    float64
 18  c18     200 non-null    float64
 19  c19     200 non-null    float64
 20  c20     200 non-null    float64
 21  c21     200 non-null    float64
dtypes: float64(22)
memory usage: 35.9 KB
    \end{Verbatim}

    \begin{tcolorbox}[breakable, size=fbox, boxrule=1pt, pad at break*=1mm,colback=cellbackground, colframe=cellborder]
\prompt{In}{incolor}{6}{\boxspacing}
\begin{Verbatim}[commandchars=\\\{\}]
\PY{n}{TrainSet}\PY{o}{.}\PY{n}{describe}\PY{p}{(}\PY{p}{)}
\end{Verbatim}
\end{tcolorbox}

            \begin{tcolorbox}[breakable, size=fbox, boxrule=.5pt, pad at break*=1mm, opacityfill=0]
\prompt{Out}{outcolor}{6}{\boxspacing}
\begin{Verbatim}[commandchars=\\\{\}]
               c0          c1          c2          c3          c4          c5  \textbackslash{}
count  200.000000  200.000000  200.000000  200.000000  200.000000  200.000000
mean    -0.704091    0.720671   -0.321631    0.728076    0.817680   -0.545805
std      2.374237    2.313069    2.807653    2.581373    2.148528    2.303483
min     -7.219070   -4.926262   -7.056635   -6.387122   -4.329674   -7.129064
25\%     -2.222519   -0.601004   -2.154710   -0.913317   -0.553274   -2.192671
50\%     -0.569562    0.499173   -0.603888    0.773361    0.631153   -0.443027
75\%      0.941912    1.892434    1.258851    2.495485    2.300333    0.697737
max      4.990854    7.784950    8.644422    7.613113    6.213770    5.714544

               c6          c7          c8          c9  {\ldots}         c12  \textbackslash{}
count  200.000000  200.000000  200.000000  200.000000  {\ldots}  200.000000
mean    -0.381539   -0.013848   -0.090832   -0.124847  {\ldots}    0.341815
std      2.200018    2.764656    2.355375    2.662448  {\ldots}    2.728096
min     -5.911245   -7.316625   -6.528675   -7.193691  {\ldots}   -6.874284
25\%     -1.805663   -1.740198   -1.786658   -1.894179  {\ldots}   -1.580786
50\%     -0.230165   -0.152530   -0.227420   -0.394961  {\ldots}    0.382955
75\%      0.949454    1.724729    1.430562    1.639940  {\ldots}    2.310996
max      6.296661    6.515889    5.646378    7.384637  {\ldots}    8.899354

              c13         c14         c15         c16         c17         c18  \textbackslash{}
count  200.000000  200.000000  200.000000  200.000000  200.000000  200.000000
mean    -0.772603    0.027777   -0.056464   -3.626522   -1.082637   -0.067145
std      2.303506    2.544115    2.641034    5.799857    5.468672    0.968266
min     -6.921811   -5.682946   -8.346379  -21.167742  -14.961184   -2.228178
25\%     -2.519280   -1.843186   -1.670472   -7.034478   -5.264901   -0.658488
50\%     -0.869128   -0.073868   -0.238759   -3.734389   -0.830931   -0.109493
75\%      0.624991    1.751848    1.936141   -0.313455    2.755951    0.594224
max      6.178711    7.142610    6.338418   14.213284   11.367014    2.957483

              c19         c20         c21
count  200.000000  200.000000  200.000000
mean    -0.031876   -0.027911    0.055325
std      1.054385    0.951471    1.033747
min     -2.753712   -2.500636   -2.886508
25\%     -0.789839   -0.685702   -0.653257
50\%     -0.131189   -0.011032    0.025438
75\%      0.735761    0.636113    0.780903
max      2.972162    2.419877    3.293263

[8 rows x 22 columns]
\end{Verbatim}
\end{tcolorbox}
        
    \begin{tcolorbox}[breakable, size=fbox, boxrule=1pt, pad at break*=1mm,colback=cellbackground, colframe=cellborder]
\prompt{In}{incolor}{7}{\boxspacing}
\begin{Verbatim}[commandchars=\\\{\}]
\PY{c+c1}{\PYZsh{}40 barras, tamaño de 15x12}
\PY{n}{TrainSet}\PY{o}{.}\PY{n}{hist}\PY{p}{(}\PY{n}{bins}\PY{o}{=}\PY{l+m+mi}{40}\PY{p}{,}\PY{n}{figsize}\PY{o}{=}\PY{p}{(}\PY{l+m+mi}{15}\PY{p}{,}\PY{l+m+mi}{12}\PY{p}{)}\PY{p}{)}
\PY{n}{plt}\PY{o}{.}\PY{n}{title}\PY{p}{(}\PY{l+s+s2}{\PYZdq{}}\PY{l+s+s2}{variables predictoras}\PY{l+s+s2}{\PYZdq{}}\PY{p}{)}
\PY{n}{plt}\PY{o}{.}\PY{n}{show}\PY{p}{(}\PY{p}{)}
\end{Verbatim}
\end{tcolorbox}

    \begin{center}
    \adjustimage{max size={0.9\linewidth}{0.9\paperheight}}{Act3_files/Act3_11_0.png}
    \end{center}
    { \hspace*{\fill} \\}
    
    \begin{tcolorbox}[breakable, size=fbox, boxrule=1pt, pad at break*=1mm,colback=cellbackground, colframe=cellborder]
\prompt{In}{incolor}{8}{\boxspacing}
\begin{Verbatim}[commandchars=\\\{\}]
\PY{c+c1}{\PYZsh{}Máximo de 40 barras}
\PY{n}{plt}\PY{o}{.}\PY{n}{hist}\PY{p}{(}\PY{n}{YTrain}\PY{p}{,}\PY{n}{bins}\PY{o}{=}\PY{l+m+mi}{40}\PY{p}{)}
\PY{n}{plt}\PY{o}{.}\PY{n}{title}\PY{p}{(}\PY{l+s+s2}{\PYZdq{}}\PY{l+s+s2}{variable de respuesta}\PY{l+s+s2}{\PYZdq{}}\PY{p}{)}
\PY{n}{plt}\PY{o}{.}\PY{n}{show}\PY{p}{(}\PY{p}{)}
\end{Verbatim}
\end{tcolorbox}

    \begin{center}
    \adjustimage{max size={0.9\linewidth}{0.9\paperheight}}{Act3_files/Act3_12_0.png}
    \end{center}
    { \hspace*{\fill} \\}
    
    \hypertarget{obtuxe9n-un-modelo-de-clasificaciuxf3n-mediante-un-uxe1rbol-de-decisiuxf3n.-utiliza-la-funciuxf3n-decisiontreeclassifier-de-la-libreruxeda-scikit-learn-utilizando-los-argumentos-por-defecto.}{%
\subsubsection{Obtén un modelo de clasificación mediante un árbol de
decisión. Utiliza la función DecisionTreeClassifier de la librería
scikit-learn utilizando los argumentos por
defecto.}\label{obtuxe9n-un-modelo-de-clasificaciuxf3n-mediante-un-uxe1rbol-de-decisiuxf3n.-utiliza-la-funciuxf3n-decisiontreeclassifier-de-la-libreruxeda-scikit-learn-utilizando-los-argumentos-por-defecto.}}

    \begin{tcolorbox}[breakable, size=fbox, boxrule=1pt, pad at break*=1mm,colback=cellbackground, colframe=cellborder]
\prompt{In}{incolor}{9}{\boxspacing}
\begin{Verbatim}[commandchars=\\\{\}]
\PY{c+c1}{\PYZsh{}Entrenamiento árbol de decisión}
\PY{n}{CLFDefault} \PY{o}{=} \PY{n}{DecisionTreeClassifier}\PY{p}{(}\PY{p}{)}
\PY{n}{CLFDefault} \PY{o}{=} \PY{n}{CLFDefault}\PY{o}{.}\PY{n}{fit}\PY{p}{(}\PY{n}{TrainSet}\PY{p}{,} \PY{n}{YTrain}\PY{p}{)}

\PY{n}{YPred} \PY{o}{=} \PY{n}{CLFDefault}\PY{o}{.}\PY{n}{predict}\PY{p}{(}\PY{n}{TestSet}\PY{p}{)}

\PY{n}{AccuracyTreeDefault} \PY{o}{=} \PY{n}{accuracy\PYZus{}score}\PY{p}{(}\PY{n}{YPred}\PY{p}{,} \PY{n}{YTest}\PY{p}{)}
\PY{n}{AccuracyTreeDefault}
\end{Verbatim}
\end{tcolorbox}

            \begin{tcolorbox}[breakable, size=fbox, boxrule=.5pt, pad at break*=1mm, opacityfill=0]
\prompt{Out}{outcolor}{9}{\boxspacing}
\begin{Verbatim}[commandchars=\\\{\}]
0.75
\end{Verbatim}
\end{tcolorbox}
        
    \begin{tcolorbox}[breakable, size=fbox, boxrule=1pt, pad at break*=1mm,colback=cellbackground, colframe=cellborder]
\prompt{In}{incolor}{22}{\boxspacing}
\begin{Verbatim}[commandchars=\\\{\}]
\PY{c+c1}{\PYZsh{}Gráfico de árbol de decisión}
\PY{n}{DotData} \PY{o}{=} \PY{n}{export\PYZus{}graphviz}\PY{p}{(}
    \PY{n}{CLFDefault}\PY{p}{,} 
    \PY{n}{out\PYZus{}file}\PY{o}{=}\PY{k+kc}{None}\PY{p}{,} 
    \PY{n}{feature\PYZus{}names}\PY{o}{=}\PY{n}{Features}\PY{p}{,}  
    \PY{n}{filled}\PY{o}{=}\PY{k+kc}{True}\PY{p}{,} 
    \PY{n}{rounded}\PY{o}{=}\PY{k+kc}{True}\PY{p}{,}  
    \PY{n}{special\PYZus{}characters}\PY{o}{=}\PY{k+kc}{True}
\PY{p}{)}  
\PY{n}{Graph} \PY{o}{=} \PY{n}{graphviz}\PY{o}{.}\PY{n}{Source}\PY{p}{(}\PY{n}{DotData}\PY{p}{)}  
\PY{n}{Graph} 
\end{Verbatim}
\end{tcolorbox}
 
            
\prompt{Out}{outcolor}{22}{}
    
    \begin{center}
    \adjustimage{max size={0.9\linewidth}{0.9\paperheight}}{Act3_files/Act3_15_0.pdf}
    \end{center}
    { \hspace*{\fill} \\}
    

    \hypertarget{ahora-indica-que-el-nivel-de-profundidad-muxe1ximo-permitido-es-de-3.-quuxe9-diferencias-observas-con-el-uxe1rbol-anterior}{%
\subsubsection{Ahora indica que el nivel de profundidad máximo permitido
es de 3. ¿Qué diferencias observas con el árbol
anterior?}\label{ahora-indica-que-el-nivel-de-profundidad-muxe1ximo-permitido-es-de-3.-quuxe9-diferencias-observas-con-el-uxe1rbol-anterior}}

    La principal diferencia es la profundidad que se le ha configurado, ya
que el primero no tenía restricciones de profundidad. En este caso la
profundidad máxima es de 3, lo cuál significa que, luego del nodo
inicial, tenemos 3 niveles más en el árbol. Al ser un árbol binario, la
máxima cantidad de nodos alcanzados en la última capa es de 8
nodos(2\^{}n). Esto significa que el árbol original, solo es considerado
hasta su tercera capa.

    \begin{tcolorbox}[breakable, size=fbox, boxrule=1pt, pad at break*=1mm,colback=cellbackground, colframe=cellborder]
\prompt{In}{incolor}{11}{\boxspacing}
\begin{Verbatim}[commandchars=\\\{\}]
\PY{c+c1}{\PYZsh{}Entrenamiento de árbol de decisión con un máximo de 3 niveles de profundidad}
\PY{n}{CLFMaxDepth3} \PY{o}{=} \PY{n}{DecisionTreeClassifier}\PY{p}{(}\PY{n}{max\PYZus{}depth}\PY{o}{=}\PY{l+m+mi}{3}\PY{p}{)}
\PY{n}{CLFMaxDepth3} \PY{o}{=} \PY{n}{CLFMaxDepth3}\PY{o}{.}\PY{n}{fit}\PY{p}{(}\PY{n}{TrainSet}\PY{p}{,} \PY{n}{YTrain}\PY{p}{)}

\PY{n}{YPred} \PY{o}{=} \PY{n}{CLFMaxDepth3}\PY{o}{.}\PY{n}{predict}\PY{p}{(}\PY{n}{TestSet}\PY{p}{)}

\PY{n}{AccuracyMaxDepth3} \PY{o}{=} \PY{n}{accuracy\PYZus{}score}\PY{p}{(}\PY{n}{YPred}\PY{p}{,} \PY{n}{YTest}\PY{p}{)}
\PY{n+nb}{print}\PY{p}{(}\PY{n}{AccuracyMaxDepth3}\PY{p}{)} 
\end{Verbatim}
\end{tcolorbox}

    \begin{Verbatim}[commandchars=\\\{\}]
0.8
    \end{Verbatim}

    \begin{tcolorbox}[breakable, size=fbox, boxrule=1pt, pad at break*=1mm,colback=cellbackground, colframe=cellborder]
\prompt{In}{incolor}{12}{\boxspacing}
\begin{Verbatim}[commandchars=\\\{\}]
\PY{c+c1}{\PYZsh{}Gráfico de árbol con un máximo de 3 niveles de profundidad}
\PY{n}{DotData} \PY{o}{=} \PY{n}{export\PYZus{}graphviz}\PY{p}{(}
    \PY{n}{CLFMaxDepth3}\PY{p}{,} 
    \PY{n}{out\PYZus{}file}\PY{o}{=}\PY{k+kc}{None}\PY{p}{,} 
    \PY{n}{feature\PYZus{}names}\PY{o}{=}\PY{n}{Features}\PY{p}{,}  
    \PY{n}{filled}\PY{o}{=}\PY{k+kc}{True}\PY{p}{,} 
    \PY{n}{rounded}\PY{o}{=}\PY{k+kc}{True}\PY{p}{,}  
    \PY{n}{special\PYZus{}characters}\PY{o}{=}\PY{k+kc}{True}
\PY{p}{)}  
\PY{n}{Graph} \PY{o}{=} \PY{n}{graphviz}\PY{o}{.}\PY{n}{Source}\PY{p}{(}\PY{n}{DotData}\PY{p}{)}  
\PY{n}{Graph} 
\end{Verbatim}
\end{tcolorbox}
 
            
\prompt{Out}{outcolor}{12}{}
    
    \begin{center}
    \adjustimage{max size={0.9\linewidth}{0.9\paperheight}}{Act3_files/Act3_19_0.pdf}
    \end{center}
    { \hspace*{\fill} \\}
    

    \hypertarget{realiza-los-siguientes-modelos-de-ensamble}{%
\subsubsection{Realiza los siguientes modelos de
ensamble:}\label{realiza-los-siguientes-modelos-de-ensamble}}

\begin{itemize}
\tightlist
\item
  Con reemplazamiento (bagging).
\item
  Sin reemplazamiento (pasting).
\item
  Realizando Random Forest (fijando el número de nodos hoja máximo a 4).
\item
  GBM.
\end{itemize}

\hypertarget{cuuxe1l-es-mejor}{%
\paragraph{¿Cuál es mejor?}\label{cuuxe1l-es-mejor}}

    \hypertarget{bagging-con-reemplazamiento}{%
\subsubsection{Bagging, con
reemplazamiento}\label{bagging-con-reemplazamiento}}

    \begin{tcolorbox}[breakable, size=fbox, boxrule=1pt, pad at break*=1mm,colback=cellbackground, colframe=cellborder]
\prompt{In}{incolor}{13}{\boxspacing}
\begin{Verbatim}[commandchars=\\\{\}]
\PY{n}{CLFBag} \PY{o}{=} \PY{n}{BaggingClassifier}\PY{p}{(}
    \PY{n}{DecisionTreeClassifier}\PY{p}{(}\PY{n}{random\PYZus{}state} \PY{o}{=} \PY{l+m+mi}{3}\PY{p}{)}\PY{p}{,} 
    \PY{n}{n\PYZus{}estimators} \PY{o}{=} \PY{l+m+mi}{100}\PY{p}{,}
    \PY{n}{max\PYZus{}samples} \PY{o}{=} \PY{l+m+mi}{70}\PY{p}{,}
    \PY{n}{bootstrap}\PY{o}{=}\PY{k+kc}{True}\PY{p}{,} 
    \PY{n}{random\PYZus{}state} \PY{o}{=} \PY{l+m+mi}{3}
\PY{p}{)}
\PY{n}{CLFBag}\PY{o}{.}\PY{n}{fit}\PY{p}{(}\PY{n}{TrainSet}\PY{p}{,} \PY{n}{YTrain}\PY{p}{)}

\PY{n}{YPred} \PY{o}{=} \PY{n}{CLFBag}\PY{o}{.}\PY{n}{predict}\PY{p}{(}\PY{n}{TestSet}\PY{p}{)}

\PY{n}{AccuracyBagging} \PY{o}{=} \PY{n}{accuracy\PYZus{}score}\PY{p}{(}\PY{n}{YTest}\PY{p}{,} \PY{n}{YPred}\PY{p}{)}
\PY{n+nb}{print}\PY{p}{(}\PY{n}{AccuracyBagging}\PY{p}{)}
\end{Verbatim}
\end{tcolorbox}

    \begin{Verbatim}[commandchars=\\\{\}]
0.9
    \end{Verbatim}

    \hypertarget{pasting-sin-reemplazamiento}{%
\subsubsection{Pasting, sin
reemplazamiento}\label{pasting-sin-reemplazamiento}}

    \begin{tcolorbox}[breakable, size=fbox, boxrule=1pt, pad at break*=1mm,colback=cellbackground, colframe=cellborder]
\prompt{In}{incolor}{14}{\boxspacing}
\begin{Verbatim}[commandchars=\\\{\}]
\PY{n}{CLFPasting} \PY{o}{=} \PY{n}{BaggingClassifier}\PY{p}{(}
    \PY{n}{DecisionTreeClassifier}\PY{p}{(}\PY{n}{random\PYZus{}state} \PY{o}{=} \PY{l+m+mi}{3}\PY{p}{)}\PY{p}{,} 
    \PY{n}{n\PYZus{}estimators} \PY{o}{=} \PY{l+m+mi}{100}\PY{p}{,}
    \PY{n}{max\PYZus{}samples} \PY{o}{=} \PY{l+m+mi}{70}\PY{p}{,} 
    \PY{n}{bootstrap}\PY{o}{=}\PY{k+kc}{False}\PY{p}{,}
    \PY{n}{random\PYZus{}state} \PY{o}{=} \PY{l+m+mi}{3}
\PY{p}{)}
\PY{n}{CLFPasting}\PY{o}{.}\PY{n}{fit}\PY{p}{(}\PY{n}{TrainSet}\PY{p}{,} \PY{n}{YTrain}\PY{p}{)}

\PY{n}{YPred} \PY{o}{=} \PY{n}{CLFPasting}\PY{o}{.}\PY{n}{predict}\PY{p}{(}\PY{n}{TestSet}\PY{p}{)}

\PY{n}{AccuracyPasting} \PY{o}{=} \PY{n}{accuracy\PYZus{}score}\PY{p}{(}\PY{n}{YTest}\PY{p}{,} \PY{n}{YPred}\PY{p}{)}
\PY{n+nb}{print}\PY{p}{(}\PY{n}{AccuracyPasting}\PY{p}{)}
\end{Verbatim}
\end{tcolorbox}

    \begin{Verbatim}[commandchars=\\\{\}]
0.85
    \end{Verbatim}

    \hypertarget{random-forest}{%
\subsubsection{Random forest}\label{random-forest}}

    \begin{tcolorbox}[breakable, size=fbox, boxrule=1pt, pad at break*=1mm,colback=cellbackground, colframe=cellborder]
\prompt{In}{incolor}{15}{\boxspacing}
\begin{Verbatim}[commandchars=\\\{\}]
\PY{n}{CLFRandomForest} \PY{o}{=} \PY{n}{RandomForestClassifier}\PY{p}{(}
    \PY{n}{n\PYZus{}estimators} \PY{o}{=} \PY{l+m+mi}{100}\PY{p}{,} 
    \PY{n}{max\PYZus{}leaf\PYZus{}nodes} \PY{o}{=} \PY{l+m+mi}{4}\PY{p}{,} 
    \PY{n}{random\PYZus{}state} \PY{o}{=} \PY{l+m+mi}{3}\PY{p}{,} 
    \PY{n}{max\PYZus{}samples} \PY{o}{=} \PY{l+m+mi}{70}
\PY{p}{)}
\PY{n}{CLFRandomForest}\PY{o}{.}\PY{n}{fit}\PY{p}{(}\PY{n}{TrainSet}\PY{p}{,} \PY{n}{YTrain}\PY{p}{)}

\PY{n}{YPred} \PY{o}{=} \PY{n}{CLFRandomForest}\PY{o}{.}\PY{n}{predict}\PY{p}{(}\PY{n}{TestSet}\PY{p}{)}

\PY{n}{AccuracyRandomForest} \PY{o}{=} \PY{n}{accuracy\PYZus{}score}\PY{p}{(}\PY{n}{YTest}\PY{p}{,} \PY{n}{YPred}\PY{p}{)}
\PY{n+nb}{print}\PY{p}{(}\PY{n}{AccuracyRandomForest}\PY{p}{)}
\end{Verbatim}
\end{tcolorbox}

    \begin{Verbatim}[commandchars=\\\{\}]
0.85
    \end{Verbatim}

    \hypertarget{gbm-gradient-boosting-classifier}{%
\subsubsection{GBM, Gradient Boosting
Classifier}\label{gbm-gradient-boosting-classifier}}

    \begin{tcolorbox}[breakable, size=fbox, boxrule=1pt, pad at break*=1mm,colback=cellbackground, colframe=cellborder]
\prompt{In}{incolor}{16}{\boxspacing}
\begin{Verbatim}[commandchars=\\\{\}]
\PY{n}{CLFGBM} \PY{o}{=} \PY{n}{GradientBoostingClassifier}\PY{p}{(}
    \PY{n}{n\PYZus{}estimators} \PY{o}{=} \PY{l+m+mi}{100}\PY{p}{,}
    \PY{n}{random\PYZus{}state} \PY{o}{=} \PY{l+m+mi}{3}
\PY{p}{)}
\PY{n}{CLFGBM}\PY{o}{.}\PY{n}{fit}\PY{p}{(}\PY{n}{TrainSet}\PY{p}{,} \PY{n}{YTrain}\PY{p}{)}

\PY{n}{YPred} \PY{o}{=} \PY{n}{CLFGBM}\PY{o}{.}\PY{n}{predict}\PY{p}{(}\PY{n}{TestSet}\PY{p}{)}

\PY{n}{AccuracyGBM} \PY{o}{=} \PY{n}{accuracy\PYZus{}score}\PY{p}{(}\PY{n}{YTest}\PY{p}{,} \PY{n}{YPred}\PY{p}{)}
\PY{n+nb}{print}\PY{p}{(}\PY{n}{AccuracyGBM}\PY{p}{)}
\end{Verbatim}
\end{tcolorbox}

    \begin{Verbatim}[commandchars=\\\{\}]
0.85
    \end{Verbatim}

    \hypertarget{cuuxe1l-muxe9todo-es-mejor}{%
\subsubsection{Cuál método es mejor?}\label{cuuxe1l-muxe9todo-es-mejor}}

    El mejor método es el más preciso, en este caso, el de mayor precisión
es ``Bagging''.

    \begin{tcolorbox}[breakable, size=fbox, boxrule=1pt, pad at break*=1mm,colback=cellbackground, colframe=cellborder]
\prompt{In}{incolor}{17}{\boxspacing}
\begin{Verbatim}[commandchars=\\\{\}]
\PY{n}{metrics} \PY{o}{=} \PY{p}{\PYZob{}}
    \PY{l+s+s1}{\PYZsq{}}\PY{l+s+s1}{Model}\PY{l+s+s1}{\PYZsq{}}\PY{p}{:}\PY{p}{[}\PY{l+s+s1}{\PYZsq{}}\PY{l+s+s1}{Bagging}\PY{l+s+s1}{\PYZsq{}}\PY{p}{,} \PY{l+s+s1}{\PYZsq{}}\PY{l+s+s1}{Pasting}\PY{l+s+s1}{\PYZsq{}}\PY{p}{,} \PY{l+s+s1}{\PYZsq{}}\PY{l+s+s1}{Random Forest}\PY{l+s+s1}{\PYZsq{}}\PY{p}{,} \PY{l+s+s1}{\PYZsq{}}\PY{l+s+s1}{GBM}\PY{l+s+s1}{\PYZsq{}}\PY{p}{]}\PY{p}{,}
    \PY{l+s+s1}{\PYZsq{}}\PY{l+s+s1}{Accuracy}\PY{l+s+s1}{\PYZsq{}}\PY{p}{:} \PY{p}{[}\PY{n}{AccuracyBagging}\PY{p}{,} \PY{n}{AccuracyPasting}\PY{p}{,} \PY{n}{AccuracyRandomForest}\PY{p}{,} \PY{n}{AccuracyGBM}\PY{p}{]}
\PY{p}{\PYZcb{}}

\PY{n+nb}{print}\PY{p}{(}\PY{n}{pd}\PY{o}{.}\PY{n}{DataFrame}\PY{p}{(}\PY{n}{metrics}\PY{p}{)}\PY{p}{)}
\end{Verbatim}
\end{tcolorbox}

    \begin{Verbatim}[commandchars=\\\{\}]
           Model  Accuracy
0        Bagging      0.90
1        Pasting      0.85
2  Random Forest      0.85
3            GBM      0.85
    \end{Verbatim}

    \hypertarget{analiza-la-importancia-de-las-variables-de-cada-uno-de-estos-muxe9todos.}{%
\subsubsection{Analiza la importancia de las variables de cada uno de
estos
métodos.}\label{analiza-la-importancia-de-las-variables-de-cada-uno-de-estos-muxe9todos.}}

    \begin{tcolorbox}[breakable, size=fbox, boxrule=1pt, pad at break*=1mm,colback=cellbackground, colframe=cellborder]
\prompt{In}{incolor}{18}{\boxspacing}
\begin{Verbatim}[commandchars=\\\{\}]
\PY{n}{ModelsNames} \PY{o}{=} \PY{p}{[}\PY{l+s+s2}{\PYZdq{}}\PY{l+s+s2}{DefaultTree}\PY{l+s+s2}{\PYZdq{}}\PY{p}{,} \PY{l+s+s2}{\PYZdq{}}\PY{l+s+s2}{Max Depth 3 Tree}\PY{l+s+s2}{\PYZdq{}}\PY{p}{,} \PY{l+s+s2}{\PYZdq{}}\PY{l+s+s2}{Bagging}\PY{l+s+s2}{\PYZdq{}}\PY{p}{,} \PY{l+s+s2}{\PYZdq{}}\PY{l+s+s2}{Pasting}\PY{l+s+s2}{\PYZdq{}}\PY{p}{,} \PY{l+s+s2}{\PYZdq{}}\PY{l+s+s2}{Random Forest}\PY{l+s+s2}{\PYZdq{}}\PY{p}{,} \PY{l+s+s2}{\PYZdq{}}\PY{l+s+s2}{GBM}\PY{l+s+s2}{\PYZdq{}}\PY{p}{]}
\PY{n}{Models} \PY{o}{=} \PY{p}{[}\PY{n}{CLFDefault}\PY{p}{,} \PY{n}{CLFMaxDepth3}\PY{p}{,} \PY{n}{CLFBag}\PY{p}{,} \PY{n}{CLFPasting}\PY{p}{,} \PY{n}{CLFRandomForest}\PY{p}{,} \PY{n}{CLFGBM}\PY{p}{]}

\PY{n}{importances} \PY{o}{=} \PY{p}{[}\PY{p}{]}
\PY{n}{importances}\PY{o}{.}\PY{n}{append}\PY{p}{(}\PY{n}{CLFDefault}\PY{o}{.}\PY{n}{feature\PYZus{}importances\PYZus{}}\PY{p}{)}
\PY{n}{importances}\PY{o}{.}\PY{n}{append}\PY{p}{(}\PY{n}{CLFMaxDepth3}\PY{o}{.}\PY{n}{feature\PYZus{}importances\PYZus{}}\PY{p}{)}

\PY{n}{importances}\PY{o}{.}\PY{n}{append}\PY{p}{(}
    \PY{n}{np}\PY{o}{.}\PY{n}{mean}\PY{p}{(}\PY{p}{[}\PY{n}{tree}\PY{o}{.}\PY{n}{feature\PYZus{}importances\PYZus{}} \PY{k}{for} \PY{n}{tree} \PY{o+ow}{in} \PY{n}{CLFBag}\PY{o}{.}\PY{n}{estimators\PYZus{}}\PY{p}{]}\PY{p}{,} 
    \PY{n}{axis}\PY{o}{=}\PY{l+m+mi}{0}\PY{p}{)}
    \PY{p}{)}
\PY{n}{importances}\PY{o}{.}\PY{n}{append}\PY{p}{(}
    \PY{n}{np}\PY{o}{.}\PY{n}{mean}\PY{p}{(}\PY{p}{[}\PY{n}{tree}\PY{o}{.}\PY{n}{feature\PYZus{}importances\PYZus{}} \PY{k}{for} \PY{n}{tree} \PY{o+ow}{in} \PY{n}{CLFPasting}\PY{o}{.}\PY{n}{estimators\PYZus{}}\PY{p}{]}\PY{p}{,} 
    \PY{n}{axis}\PY{o}{=}\PY{l+m+mi}{0}\PY{p}{)}
    \PY{p}{)}
\PY{n}{importances}\PY{o}{.}\PY{n}{append}\PY{p}{(}\PY{n}{CLFRandomForest}\PY{o}{.}\PY{n}{feature\PYZus{}importances\PYZus{}}\PY{p}{)}
\PY{n}{importances}\PY{o}{.}\PY{n}{append}\PY{p}{(}\PY{n}{CLFGBM}\PY{o}{.}\PY{n}{feature\PYZus{}importances\PYZus{}}\PY{p}{)}
\end{Verbatim}
\end{tcolorbox}

    \begin{tcolorbox}[breakable, size=fbox, boxrule=1pt, pad at break*=1mm,colback=cellbackground, colframe=cellborder]
\prompt{In}{incolor}{19}{\boxspacing}
\begin{Verbatim}[commandchars=\\\{\}]
\PY{n}{Features} \PY{o}{=} \PY{n}{np}\PY{o}{.}\PY{n}{array}\PY{p}{(}\PY{n}{Features}\PY{p}{)}

\PY{n}{fig}\PY{p}{,} \PY{n}{axs} \PY{o}{=} \PY{n}{plt}\PY{o}{.}\PY{n}{subplots}\PY{p}{(}
    \PY{n}{nrows}\PY{o}{=}\PY{n+nb}{len}\PY{p}{(}\PY{n}{ModelsNames}\PY{p}{)}\PY{p}{,} 
    \PY{n}{figsize}\PY{o}{=}\PY{p}{(}\PY{l+m+mi}{10}\PY{p}{,}\PY{l+m+mi}{10}\PY{p}{)}
    \PY{p}{)}
\PY{n}{fig}\PY{o}{.}\PY{n}{suptitle}\PY{p}{(}\PY{l+s+s2}{\PYZdq{}}\PY{l+s+s2}{Variables Importance}\PY{l+s+s2}{\PYZdq{}}\PY{p}{)}

\PY{k}{for} \PY{n}{i}\PY{p}{,}\PY{n}{modelo} \PY{o+ow}{in} \PY{n+nb}{enumerate}\PY{p}{(}\PY{n}{Models}\PY{p}{)}\PY{p}{:}
    \PY{n}{order\PYZus{}index} \PY{o}{=} \PY{n}{np}\PY{o}{.}\PY{n}{argsort}\PY{p}{(}\PY{n}{importances}\PY{p}{[}\PY{n}{i}\PY{p}{]}\PY{p}{)}\PY{p}{[}\PY{p}{:}\PY{p}{:}\PY{o}{\PYZhy{}}\PY{l+m+mi}{1}\PY{p}{]}
    \PY{n}{sns}\PY{o}{.}\PY{n}{barplot}\PY{p}{(}
        \PY{n}{x}\PY{o}{=}\PY{n}{Features}\PY{p}{[}\PY{n}{order\PYZus{}index}\PY{p}{]}\PY{p}{,}
        \PY{n}{y}\PY{o}{=}\PY{n}{importances}\PY{p}{[}\PY{n}{i}\PY{p}{]}\PY{p}{[}\PY{n}{order\PYZus{}index}\PY{p}{]}\PY{p}{,}
        \PY{n}{ax}\PY{o}{=}\PY{n}{axs}\PY{p}{[}\PY{n}{i}\PY{p}{]}
        \PY{p}{)}

    \PY{n}{axs}\PY{p}{[}\PY{n}{i}\PY{p}{]}\PY{o}{.}\PY{n}{set\PYZus{}title}\PY{p}{(}\PY{n}{ModelsNames}\PY{p}{[}\PY{n}{i}\PY{p}{]}\PY{p}{)}
    \PY{n}{axs}\PY{p}{[}\PY{n}{i}\PY{p}{]}\PY{o}{.}\PY{n}{set\PYZus{}xlabel}\PY{p}{(}\PY{l+s+s2}{\PYZdq{}}\PY{l+s+s2}{Variables}\PY{l+s+s2}{\PYZdq{}}\PY{p}{)}
    \PY{n}{axs}\PY{p}{[}\PY{n}{i}\PY{p}{]}\PY{o}{.}\PY{n}{set\PYZus{}ylabel}\PY{p}{(}\PY{l+s+s2}{\PYZdq{}}\PY{l+s+s2}{Importance}\PY{l+s+s2}{\PYZdq{}}\PY{p}{)}
    
\PY{n}{plt}\PY{o}{.}\PY{n}{subplots\PYZus{}adjust}\PY{p}{(}\PY{n}{hspace}\PY{o}{=}\PY{l+m+mi}{1}\PY{p}{)}
\PY{n}{plt}\PY{o}{.}\PY{n}{show}\PY{p}{(}\PY{p}{)} 
\end{Verbatim}
\end{tcolorbox}

    \begin{center}
    \adjustimage{max size={0.9\linewidth}{0.9\paperheight}}{Act3_files/Act3_34_0.png}
    \end{center}
    { \hspace*{\fill} \\}
    
    \begin{tcolorbox}[breakable, size=fbox, boxrule=1pt, pad at break*=1mm,colback=cellbackground, colframe=cellborder]
\prompt{In}{incolor}{ }{\boxspacing}
\begin{Verbatim}[commandchars=\\\{\}]

\end{Verbatim}
\end{tcolorbox}


    % Add a bibliography block to the postdoc
    
    
    
\end{document}
